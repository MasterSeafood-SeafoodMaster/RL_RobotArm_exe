\documentclass[class=NCU_thesis, crop=false]{standalone}
\begin{document}

\chapter{總結}

\section{結論}

本研究分別嘗試了在不同場景中的機械臂應用,並分析了其控制方法和結果。在實驗中,通過設計機械結構、撰寫控制程式和協調大型語言模型,在進行多次測試後,成功地展示了機械臂在基本控制、畫圖和自動運輸車上的應用。

在第一個實驗中,我們探索了機械臂的基本控制,通過簡化指令格式和優化控制函數,使得機械臂能夠精確地完成指定任務。實驗結果顯示,此方法在提高指令生成效率和準確性方面具有顯著效果。

第二個實驗中,我們將機械臂用於畫圖,成功地實現了複雜圖形的繪製。這展示了機械臂在精細操作方面的潛力,並為其在工業設計和藝術創作中的應用提供了可能性。

第三個實驗探討了機械臂在自動運輸車上的應用。實驗結果表明,儘管在硬體連動過程中存在一定的機械性失誤,但通過進一步優化硬體間的協調機制,可以顯著提升系統的穩定性和精確度。

\section{未來展望}
在未來的研究中,我們計劃進一步優化機械臂的控制系統,尤其是在硬體協調方面。此外,將更多的人工智慧技術應用於機械臂控制,探索其在更複雜任務中的潛力,例如自動化生產線和高精度醫療操作。

總之,本研究為機械臂技術的進一步發展奠定了堅實的基礎,並展示了其在多種應用場景中的廣闊前景。
\end{document}