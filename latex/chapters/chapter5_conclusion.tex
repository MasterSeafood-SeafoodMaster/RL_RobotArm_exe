\documentclass[class=NCU_thesis, crop=false]{standalone}
\begin{document}

\chapter{總結}

\section{結論}

本研究分別嘗試了在不同場景中的機械臂應用,並分析了其控制方法和結果。在實驗中,通過設計機械結構、撰寫控制程式和協調大型語言模型,在進行多次測試後,成功地展示了機械臂在基本控制、畫圖和自動運輸車上的應用。

在第一個實驗中,我們探索了機械臂的基本控制,通過簡化指令格式和優化控制函數,使得機械臂能夠精確地完成指定任務。實驗結果顯示,此方法在提高指令生成效率和準確性方面具有顯著效果。

第二個實驗中,我們將機械臂用於畫圖,成功地實現了複雜圖形的繪製。這展示了機械臂在精細操作方面的潛力,並為其在工業設計和藝術創作中的應用提供了可能性。

第三個實驗探討了機械臂在自動運輸車上的應用。實驗結果表明,儘管在硬體連動過程中存在一定的機械性失誤,但通過進一步優化硬體間的協調機制,可以顯著提升系統的穩定性和精確度。

\section{未來展望}

在未來的研究中,我們希望進一步優化機械臂的控制系統,尤其是在硬體協調方面。我們將探索更便利的互動方式,例如通過語音指令直接控制機械臂,使操作更加直觀和方便。讓使用者能夠以自然的語言與機械臂進行溝通和控制,從而大大降低操作的複雜性和學習成本。

此外,我們還計劃利用3D列印技術來製作更多客製化的機械臂及其零部件,並協調整合更多的硬體設備。使我們能夠快速設計和生產符合特定應用需求的機械臂,從而滿足不同行業和應用場景的需求。

另外,我們將研究和開發更多模組化的機械臂控制函式庫,以便大型語言模型能夠直接使用這些函式庫進行機械臂的控制。這些模組化函式庫將包含多種預先定義的操作和功能,使得開發者可以輕鬆地構建和定制機械臂的操作流程,進一步提升開發效率和彈性。

總之,本研究展示了智慧機器與大型語言模型在多種應用場景中的廣闊前景。我們期待未來能夠通過技術創新和多設備整合,進一步擴展大型語言模型在機器控制上的應用範圍,推動機械臂技術向更加智能化、人性化和客製化的方向發展。
\end{document}