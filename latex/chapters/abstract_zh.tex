\documentclass[class=NCU_thesis, crop=false]{standalone}
\begin{document}

\chapter{摘要}

本研究探討了如何透過大型語言模型(Large Language Model, LLM),將自然語言轉換為程式碼來控制機器。研究內容涵蓋背景知識、文獻回顧、研究方法、以及實驗設計與結果。

在背景知識和文獻回顧部分,首先介紹了大型語言模型的研究現況、智慧機器與人工智慧物聯網的應用場景,以及3D列印技術的發展現況。接著,回顧了大型語言模型在程式碼生成與機器控制上的應用、運動學研究與機器人控制的相關文獻,以及3D列印在智慧機器中的應用場景。

研究方法部分描述了硬體設計流程,包括模型設計軟體Autodesk Fusion 360、檔案輸出格式STL、3D列印機Creality K1 MAX,以及馬達與開發版的介紹。運動學開發方面,涵蓋了運動模擬環境、順向運動學和逆向運動學的開發。大型語言模型的選擇包括OpenAI的介紹和下prompt的規則。最後,系統架構章節詳細介紹了系統架構圖、client端的輸入內容以及轉換到程式碼的過程。

實驗設計與結果部分包含三個實驗。分別為機械臂的基本控制、機械臂應用於畫圖與機械臂在自動運輸車上的應用,其中展示了機械結構設計圖、函數設計、下達指令的格式以及實驗結果。

實驗結果顯示,使用大型語言模型生成程式碼來控制機器的方式,不僅提高了控制指令的生成效率和準確性,也展示了在各種應用場景中的可行性。然而,隨著硬體的增多和系統的複雜性增加,也面臨一些機械性失誤的挑戰。未來的研究將繼續優化系統的穩定性和精確度,進一步提升其應用價值。

\vspace{2em}
\noindent \textbf{關鍵字:自然語言、大型語言模型、程式碼生成、3D列印、智慧機器} \keywordsZh{} % Set keywords in config.tex
\end{document}