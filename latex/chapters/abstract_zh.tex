\documentclass[class=NCU_thesis, crop=false]{standalone}
\begin{document}

\chapter{摘要}

本研究探討了如何透過大型語言模型(Large Language Model, LLM),將自然語言轉換為程式碼來控制機器。研究內容涵蓋背景知識、文獻回顧、研究方法、以及實驗設計與結果。

在背景知識和文獻回顧部分,首先介紹了大型語言模型的研究現況、智慧機器與人工智慧物聯網的應用場景,以及3D列印技術的發展現況。接著,回顧了大型語言模型在程式碼生成與機器控制上的應用、運動學研究與機器人控制的相關文獻,以及3D列印近年來的發展與其應用場景。

研究方法部分描述了硬體設計流程,包括模型設計軟體、檔案輸出格式、3D列印的使用,以及馬達與開發版的介紹。軟體設計流程方面,介紹了運動模擬環境、順向運動學和逆向運動學的開發,與大型語言模型應用程式介面的使用,最後,系統架構章節詳細介紹了系統架構圖、系統流程圖等整體程式框架。

實驗設計與結果部分包含三個實驗。分別為機械臂的基本控制、機械臂應用於畫圖與機械臂在自動運輸車上的應用,其中展示了機械結構設計圖、函數設計、下達指令的格式、實驗過程縮圖以及最後的實驗成效總結。

而實驗結果顯示,使用大型語言模型生成程式碼來控制機器的方式擁有相當高的準確度,尤其在有較明確的機器函式庫的前提下,更能透過少量的指令輸入,獲得高品質生成效率和準確度。然而,隨著硬體的增多和系統的複雜性增加,也面臨了一些需要克服的機械性失誤。未來的研究將繼續優化系統的穩定性和精確度,進一步提升其應用價值。

\vspace{2em}
\noindent \textbf{關鍵字:} \keywordsZh{} % Set keywords in config.tex
\end{document}