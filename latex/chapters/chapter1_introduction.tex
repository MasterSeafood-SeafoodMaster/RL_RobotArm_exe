\documentclass[class=NCU_thesis, crop=false]{standalone}
\begin{document}

\chapter{緒論}
\section{研究動機}

在現今蓬勃發展的人工智慧領域中,
隨著高效能運算技術(High Performance Computing, HPC)~\cite{reed2022reinventing}的快速發展,
許多大型語言模型(Large Language Model, LLM)~\cite{zhao2023survey},
如OpenAI的Chat-GPT~\cite{Liu_2023}、Meta的Llama2~\cite{touvron2023llama}、Google的Gemini~\cite{geminiteam2024gemini}的問世,
至今已經深刻改變了我們對於人工智慧與其應用前景的認知。
這些大型模型的出現不僅擴大了我們對於人工智慧應用的想像,
同時突破了人工智慧技術在各個不同領域中應用的可能,
目前大型語言模型已被廣泛的應用在客服、教育、編輯寫作、程式開發與多媒體創作領域,成為了人們生活與工作中的一大幫助。
此外,對於這些模型的需求不斷增長,
也促使了各大科技公司不斷投入資源,
提供更加強大且多樣化的大型語言模型,
使的目前的人工智慧發展方向更加明朗。

本研究旨在探索大型語言模型在自動程式碼生成(Code Generation)~\cite{10196869}領域的能力,
特別是在控制機器方面的應用。
眾所周知,目前許多研究集中於微調模型本身,
以提升其在單一任務上的效能。
然而我們認為,將大型語言模型應用於控制機器,
是一個極具挑戰性且有前景的研究方向。
通過將人類的自然語言指令轉換為程式碼,
機器可以更加靈活地被控制,
進一步擴大了人與機器之間的交互性,
並使得使用者能夠以更加直觀和自然的方式與機器進行互動。

這項研究的重要性在於其對於智慧機器技術發展和應用的潛在影響。
首先,此研究不但可以大幅降低機器在軟體層面的開發與維護成本,從而推動智慧機器技術的普及和應用。
更重要的是,利用大型語言模型能很好的理解自然語言指令的特性,我們可以製作出更加貼近個人需求,
且互動性更高的智慧機器,
從而滿足不同使用者的個人化需求,
並將這些機器廣泛應用於人的生活和工作中,
為人們帶來更大的便利和效益。

\section{研究目的}

本研究旨在探索大型語言模型生成程式碼的潛力,與在控制機器方面的應用。
本研究預計達成以下目標:
\begin{itemize}

	\item 探索大型語言模型在自動程式碼生成中的應用,並達到使用自然語言控制機器的目的。

	\item 驗證使用大型語言模型將自然語言指令轉換為程式碼的可行性和實際應用效果。

	\item 自行組裝設計機器,並透過3D列印技術實現更為客製化的機型製作。

	\item 分析和評估所開發的系統在不同應用場景下的實際效果和應用價值。

	\item 提供相關技術和方法的研究成果,為自然語言控制機器的相關研究和應用提供實證基礎和技術範本。
\end{itemize}

\section{論文架構}
本論文分為五個章節,其架構如下:

第一章、緒論,敘述本論文之研究目的、動機以及架構。

第二章、背景知識以及文獻回顧,
敘述本研究之背景知識如大型語言模型的研究現況、智慧機器與人工智慧物聯網的應用場景、以及融合 3D 列印技術的機器設計,
並探討目前已有的相關研究。

第三章、研究方法,
說明本研究細節,如模型設計圖、開發板與硬體的使用、機器運動學開發、與大型語言模型的連動等。

第四章、實驗設計與結果,
展示機器的實際運作結果,產生的程式碼與準確率等資訊。

第五章、總結,
對於研究結果進行總結,並討論研究的未來展望。

\end{document}
