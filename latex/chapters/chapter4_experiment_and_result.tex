\documentclass[class=NCU_thesis, crop=false]{standalone}
\usepackage[newfloat]{minted}
\usepackage{floatrow}
\usepackage{graphicx}


\begin{document}

\chapter{實驗設計與結果}

\section{實驗一:機械臂的基本控制}
\subsection{機械結構設計圖}
本實驗的硬體部分使用了3D列印技術,結合小型伺服馬達,設計了一個頂端為夾爪的小型機械臂,以下為此硬體的詳細設計圖紙:
\begin{figure}[htbp]
    \centering
    \includegraphics[width=0.9\textwidth]{figures/Armv1 (1).PNG}
    \caption{機械臂版本一設計圖紙 第一頁(單位:mm)}
    %\label{fig:Armv1Drawing_p1}}
\end{figure}

\begin{figure}[htbp]
    \centering
    \includegraphics[width=0.9\textwidth]{figures/Armv1 (2).PNG}
    \caption{機械臂版本一設計圖紙 第二頁(單位:mm)}
\end{figure}

\begin{figure}[htbp]
    \centering
    \includegraphics[width=0.9\textwidth]{figures/Armv1 (3).PNG}
    \caption{機械臂版本一設計圖紙 第三頁(單位:mm)}
\end{figure}

\begin{figure}[htbp]
    \centering
    \includegraphics[width=0.9\textwidth]{figures/Armv1 (4).PNG}
    \caption{機械臂版本一設計圖紙 第四頁(單位:mm)}
\end{figure}

\begin{figure}[htbp]
    \centering
    \includegraphics[width=0.9\textwidth]{figures/Armv1 (5).PNG}
    \caption{機械臂版本一設計圖紙 第五頁(單位:mm)}
\end{figure}

\subsection{函數設計}
在這個實驗中,我們的目標是讓機械臂能夠成功夾取特定物體,並放置於指定位置。
這個過程涉及機械臂路徑的規劃和精確的抓取動作。
為了實現這一目標,我們撰寫了一個簡易的路徑規劃程式,
利用語言模型在填空方面的優勢,使其生成和編輯機械臂的運動指令。
使用這樣的方式,嘗試讓大型語言模型自動生成適當的路徑規劃,並將其應用於機械臂的控制,
以下為此實驗使用的程式碼:

\begin{listing}
    \begin{minted}[frame=single,
                   framesep=3mm,
                   linenos=true,
                   xleftmargin=21pt,
                   tabsize=4]{python}

        from ArmEnv_3D_ori import ArmEnv_3D
        import math

        Arm = ArmEnv_3D([5.942, 6.81, 8.747, 7.523], False)
        path = []
        #Please planning path form hear
        #path+=Arm.moveto([0, 0, 25], 20) explain: Move to [0, 0, 25] in 20 steps
        #path+=[[-91, 0, 0, 0]] explain: Close claw
        #path+=[[91, 0, 0, 0]] explain: Open claw

        #reset
        end=path[-2].copy()
        while not sum(end)==0:
            for i in range(len(end)):
                if end[i]>0: end[i]-=1
                elif end[i]<0: end[i]+=1
            path.append(end.copy())

    \end{minted}
\caption{實驗程式碼} 
\end{listing}

\subsection{下達指令的格式範例}
\begin{listing}
    \begin{minted}[frame=single,
                   framesep=3mm,
                   linenos=true,
                   xleftmargin=21pt,
                   tabsize=4]{js}
    {     
        role: "user",
        content : 
    "\
    from ArmEnv_3D_ori import ArmEnv_3D\
    import math\
    \
    Arm = ArmEnv_3D([5.942, 6.81, 8.747, 7.523], False)\
    path = []\
    #Please planning path form hear\
    #path+=Arm.moveto([0, 0, 25], 20) explain: Move to [0, 0, 25] in 20 steps\
    #path+=[[-91, 0, 0, 0]] explain: Close claw\
    #path+=[[91, 0, 0, 0]] explain: Open claw\
    \
    #reset\
    end=path[-2].copy()\
    while not sum(end)==0:\
        for i in range(len(end)):\
            if end[i]>0: end[i]-=1\
            elif end[i]<0: end[i]+=1\
        path.append(end.copy())\
    "
    },
    {
        role: "user", 
        content: "This is a sample program for controlling a robotic arm.\
                  Please modify it to help me complete the following requirements: \
                  the ball is at [10.5, 12, 2], and the top of the tower is at [9.5, 0, 14.5]. \
                  Please help me catch the ball to the top of the tower. \
                  During the process, please do not get too close to the side of the tower \
                  (keep outside the radius 7).\
                  return only code.\
                  "
    }
    \end{minted}
\caption{實驗一:指令格式範例} 
\end{listing}

\begin{listing}
    \begin{minted}[frame=single,
                   framesep=3mm,
                   linenos=true,
                   xleftmargin=21pt,
                   tabsize=4]{js}

    {
        role="assistant",
        content=
    "\
    from ArmEnv_3D_ori import ArmEnv_3D\
    import math\
    \
    Arm = ArmEnv_3D([5.942, 6.81, 8.747, 7.523], False)\
    path = []\
    path += Arm.moveto([0, 0, 25], 20)\
    path += Arm.moveto([10.5, 12, 2], 20)\
    path+=[[-91, 0, 0, 0]]\
    path += Arm.moveto([10, 13, 15], 20)\
    path += Arm.moveto([9.5, 0, 14.5], 20)\
    path+=[[91, 0, 0, 0]]\
    \
    #reset\
    end=path[-2].copy()\
    while not sum(end)==0:\
        for i in range(len(end)):\
            if end[i]>0: end[i]-=1\
            elif end[i]<0: end[i]+=1\
        path.append(end.copy())\
    "
    }

    \end{minted}
\caption{實驗一:回傳格式範例} 
\end{listing}

\subsection{實驗結果}
將球放於塔上: 7/10\\
\begin{figure}[!hbt]
    \centering
    \subcaptionbox
        {初始位置
        \label{fig:fig-dataset-contrast-after-adjustment}}
        {\includegraphics[width=0.4\linewidth]{figures/TB (1).jpg}}
    ~    
    \subcaptionbox
        {移動至黑球位置
        \label{fig:fig-dataset-contrast-after-adjustment}}
        {\includegraphics[width=0.4\linewidth]{figures/TB (2).jpg}}
    ~
    \subcaptionbox
        {夾取黑球
        \label{fig:fig-dataset-contrast-after-adjustment}}
        {\includegraphics[width=0.4\linewidth]{figures/TB (3).jpg}}
    ~
    \subcaptionbox
        {移動至塔頂位置
        \label{fig:fig-dataset-contrast-after-adjustment}}
        {\includegraphics[width=0.4\linewidth]{figures/TB (4).jpg}}
    ~    
    \subcaptionbox
        {放開黑球
        \label{fig:fig-dataset-contrast-after-adjustment}}
        {\includegraphics[width=0.4\linewidth]{figures/TB (5).jpg}}
    ~
    \subcaptionbox
        {回到初始位置
        \label{fig:fig-dataset-contrast-after-adjustment}}
        {\includegraphics[width=0.4\linewidth]{figures/TB (6).jpg}}   
\caption{實驗一:實驗過程縮圖}
\end{figure}


\section{實驗二:將機械臂用於畫圖}
\subsection{機械結構設計圖}
本實驗的硬體部分使用了3D列印技術,結合小型伺服馬達,
設計了一個畫筆的小型機械臂,以下為此硬體的詳細設計圖紙:
\begin{figure}[htbp]
    \centering
    \includegraphics[width=0.9\textwidth]{figures/Armv2 (1).PNG}
    \caption{機械臂版本二設計圖紙 第一頁(單位:mm)}
    %\label{fig:Armv1Drawing_p1}}
\end{figure}

\begin{figure}[htbp]
    \centering
    \includegraphics[width=0.9\textwidth]{figures/Armv2 (2).PNG}
    \caption{機械臂版本二設計圖紙 第二頁(單位:mm)}
\end{figure}

\begin{figure}[htbp]
    \centering
    \includegraphics[width=0.9\textwidth]{figures/Armv2 (3).PNG}
    \caption{機械臂版本二設計圖紙 第三頁(單位:mm)}
\end{figure}

\begin{figure}[htbp]
    \centering
    \includegraphics[width=0.9\textwidth]{figures/Armv2 (4).PNG}
    \caption{機械臂版本二設計圖紙 第四頁(單位:mm)}
\end{figure}

\begin{figure}[htbp]
    \centering
    \includegraphics[width=0.9\textwidth]{figures/Armv2 (5).PNG}
    \caption{機械臂版本二設計圖紙 第五頁(單位:mm)}
\end{figure}

\subsection{函數設計}
從實驗一的基礎上,我們改進了機械臂的路徑規劃方法,將其應用於讓機械臂進行畫畫。
此次實驗中,我們簡化了路徑規劃,使大型語言模型在平面座標內進行規劃,
這樣可以降低複雜度,並節省資源。通過這種簡化的路徑規劃方法,
我們能夠更有效地控制機械臂,讓其能夠在紙上或畫布上繪製出預定的圖案。
本實驗將此機器的動作指令簡化為以下函式:

\begin{itemize}
    \item draw(from, to): 由變數from提供的座標,畫至變數to提供的座標。
\end{itemize}

\subsection{下達指令的格式範例}
\begin{listing}
    \begin{minted}[frame=single,
                   framesep=3mm,
                   linenos=true,
                   xleftmargin=21pt,
                   tabsize=4]{js}
    {     
        role: "user",
        content : 
        "The following functions are available:\
        \
        draw(from, to): Let the arm draw a straight line from from to to,\
        such as draw([0, 0], [5, 5]).\
        the limit is[0, 5]\
        \
        Please help me use the above functions to control the robot arm,\
        and do not output other text other than the above functions.\
        (Use "," to separate each step)"
    },
    {
        role: "user", 
        content: "Task: Draw a square."
    }
    \end{minted}
\caption{實驗二:指令格式範例} 
\end{listing}

\begin{listing}
    \begin{minted}[frame=single,
                   framesep=3mm,
                   linenos=true,
                   xleftmargin=21pt,
                   tabsize=4]{js}

    {
        role="assistant",
        content="draw([1, 1], [1, 4]),\
                 draw([1, 4], [4, 4]),\
                 draw([4, 4], [4, 1]),\
                 draw([4, 1], [1, 1])"
    }

    \end{minted}
\caption{實驗二:回傳格式範例} 
\end{listing}

\subsection{實驗結果}
GPT3.5 方塊(10/10)、三角形(10/10)、星形(10/5)、房子(10/0)\\
GPT4o 方塊(10/10)、三角形(10/10)、星形(10/9)、房子(10/7)

\begin{figure}[!hbt]
    \centering
    \subcaptionbox
        {執行畫方形指令
        \label{fig:fig-dataset-contrast-after-adjustment}}
        {\includegraphics[width=0.4\linewidth]{figures/square_0.jpg}}
    ~    
    \subcaptionbox
        {完成畫方形指令
        \label{fig:fig-dataset-contrast-after-adjustment}}
        {\includegraphics[width=0.4\linewidth]{figures/square_1.jpg}}
    ~
    \subcaptionbox
        {執行畫三角形指令
        \label{fig:fig-dataset-contrast-after-adjustment}}
        {\includegraphics[width=0.4\linewidth]{figures/triangle_0.jpg}}
    ~
    \subcaptionbox
        {完成畫三角形指令
        \label{fig:fig-dataset-contrast-after-adjustment}}
        {\includegraphics[width=0.4\linewidth]{figures/triangle_1.jpg}}
    ~    
    \subcaptionbox
        {執行畫星形指令
        \label{fig:fig-dataset-contrast-after-adjustment}}
        {\includegraphics[width=0.4\linewidth]{figures/star_0.jpg}}
    ~
    \subcaptionbox
        {完成畫星形指令
        \label{fig:fig-dataset-contrast-after-adjustment}}
        {\includegraphics[width=0.4\linewidth]{figures/star_1.jpg}}   
\caption{實驗二:實驗過程縮圖}
\end{figure}


\section{實驗三:機械臂在自動運輸車上的應用}
\subsection{機械結構設計圖}
本實驗將機械臂與自動運輸車、鏡頭等硬體結合,設計了一台裝載機械臂的無人搬運車,以下為此硬體的詳細設計圖紙:
\begin{figure}[htbp]
    \centering
    \includegraphics[width=0.9\textwidth]{figures/Armv3 (1).PNG}
    \caption{機械臂版本三設計圖紙 第一頁(單位:mm)}
    %\label{fig:Armv1Drawing_p1}}
\end{figure}

\begin{figure}[htbp]
    \centering
    \includegraphics[width=0.9\textwidth]{figures/Armv3 (2).PNG}
    \caption{機械臂版本三設計圖紙 第二頁(單位:mm)}
\end{figure}

\begin{figure}[htbp]
    \centering
    \includegraphics[width=0.9\textwidth]{figures/Armv3 (3).PNG}
    \caption{機械臂版本三設計圖紙 第三頁(單位:mm)}
\end{figure}

\begin{figure}[htbp]
    \centering
    \includegraphics[width=0.9\textwidth]{figures/Armv3 (4).PNG}
    \caption{機械臂版本三設計圖紙 第四頁(單位:mm)}
\end{figure}

\begin{figure}[htbp]
    \centering
    \includegraphics[width=0.9\textwidth]{figures/Armv3 (5).PNG}
    \caption{機械臂版本三設計圖紙 第五頁(單位:mm)}
\end{figure}

\begin{figure}[htbp]
    \centering
    \includegraphics[width=0.9\textwidth]{figures/Armv3 (6).PNG}
    \caption{機械臂版本三設計圖紙 第六頁(單位:mm)}
\end{figure}


\subsection{函數設計}
在前兩個實驗的基礎上,我們進一步改進了控制流程,將機械臂與自動運輸車整合在一起。
這次,我們將控制函式進行細分,並模組化,使得大型語言模型能夠通過簡單的指令來控制機械臂和車輛的行動。
這種模組化的設計不僅降低了指令的複雜度,還減少了資源利用。
最終,我們能夠通過簡單的模組化指令來控制整個系統,
使得機械臂和車輛能夠協同工作,完成更複雜的任務。

\begin{itemize}
    \item find(color): 配合相機定位,驅動輪子尋找動應顏色的方塊。

    \item aim(color): 配合相機定位,驅動機械臂瞄準對應顏色的方塊。

    \item grab(): 驅動機械臂,抓取正下方的物品。

    \item reset(): 驅動機械臂,回到初始位置
\end{itemize}

\newpage
\subsection{下達指令的格式範例}
以下為指令格式:
\begin{listing}
    \begin{minted}[frame=single,
                   framesep=3mm,
                   linenos=true,
                   xleftmargin=21pt,
                   tabsize=4]{js}
    {     
        role: "user",
        content : 
        "The following functions are available:\
        \
        find(color): Let the robot look for a block of a specific color.\
        color option: red, blue.\
        aim(color): Let the robot aim at a block of a specific color.\
        color option: red, blue\
        grab: Make the robot arm grab the block and put it down.\
        reset: Return the robotic arm to its initial position\
        (needs to be executed before each aiming).\
        \
        Please help me use the above functions to control the robot arm,\
        and do not output other text other than the above functions.\
        (Use "," to separate each step)"
    },
    {
        role: "user", 
        content: "Task: Please grab the red block, and then grab the blue block."
    }
    \end{minted}
\caption{實驗三:指令格式範例} 
\end{listing}

\begin{listing}
    \begin{minted}[frame=single,
                   framesep=3mm,
                   linenos=true,
                   xleftmargin=21pt,
                   tabsize=4]{js}

    {
        role="assistant",
        content="find(red), aim(red), grab(), reset(), find(blue), aim(blue), grab()"
    }

    \end{minted}
\caption{實驗三:回傳格式範例} 
\end{listing}

\subsection{實驗結果}
夾取藍色->夾取紅色(10/8)、\\
失敗的兩次:\\
鏡頭判斷失準(瞄歪了)。


\begin{figure}[!hbt]
    \centering
    \subcaptionbox
        {瞄準紅色方塊
        \label{fig:fig-dataset-contrast-after-adjustment}}
        {\includegraphics[width=0.4\linewidth]{figures/Exp3 (2)_square.jpg}}
    ~    
    \subcaptionbox
        {夾取紅色方塊
        \label{fig:fig-dataset-contrast-after-adjustment}}
        {\includegraphics[width=0.4\linewidth]{figures/Exp3 (3)_square.jpg}}
    ~
    \subcaptionbox
        {夾取成功
        \label{fig:fig-dataset-contrast-after-adjustment}}
        {\includegraphics[width=0.4\linewidth]{figures/Exp3 (4)_square.jpg}}
    ~
    \subcaptionbox
        {瞄準藍色方塊
        \label{fig:fig-dataset-contrast-after-adjustment}}
        {\includegraphics[width=0.4\linewidth]{figures/Exp3 (5)_square.jpg}}
    ~    
    \subcaptionbox
        {夾取藍色方塊
        \label{fig:fig-dataset-contrast-after-adjustment}}
        {\includegraphics[width=0.4\linewidth]{figures/Exp3 (6)_square.jpg}}
    ~
    \subcaptionbox
        {夾取成功
        \label{fig:fig-dataset-contrast-after-adjustment}}
        {\includegraphics[width=0.4\linewidth]{figures/Exp3 (7)_square.jpg}}   
\caption{實驗三:實驗過程縮圖}
\end{figure}

\end{document}