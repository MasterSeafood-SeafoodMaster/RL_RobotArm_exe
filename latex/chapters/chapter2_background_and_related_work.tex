\documentclass[class=NCU_thesis, crop=false]{standalone}
\begin{document}

\chapter{背景知識以及文獻回顧}

\section{背景知識}
\subsection{大型語言模型的研究現況}

歷史回顧:自然語言處理的歷史中,大型語言模型的發展經歷了多個階段。從早期的統計模型到現代的深度學習模型,我們可以追溯到不同時期的重要里程碑,例如神經語言模型(NLM)、Transformer模型和BERT(Bidirectional Encoder Representations from Transformers)等。

目前現況:近年來,大型語言模型取得了巨大的進展。GPT-3、T5(Text-to-Text Transfer Transformer)和XLNet等模型在自然語言理解、生成和翻譯等任務上取得了令人矚目的成果。這些模型的規模和性能不斷提高,並且被廣泛應用於各個領域。

未來發展方向:大型語言模型的未來發展方向包括模型的更好可解釋性、更高效的訓練方法、更好的領域適應能力以及更好的多語言處理能力。此外,我們需要關注模型的公平性、隱私保護和環境影響等問題。

\subsection{智慧機器與AIOT的應用場景}

工業自動化:機械臂在製造業中的應用越來越普遍,例如組裝、焊接、物料處理等。智慧工廠中的自動化流程也依賴於機器人技術。

物流和倉儲:智慧小車、AGV(自動導引車)等在物流和倉儲管理中發揮著關鍵作用。它們可以自動運送物品,提高效率並減少人力成本。

醫療保健:機器人在手術、康復治療、輔助生活等方面有應用。例如,手術機器人可以幫助外科醫生進行精確的手術。

\subsection{融合3D列印技術的機器設計}

定制化設計:3D列印技術使機器人的部件可以根據特定需求進行定制。這種靈活性使得機器人可以更好地適應不同的應用場景,例如醫療機器人、工業機器人和家庭機器人。

輕量化和結構優化:3D列印技術可以實現複雜形狀的部件製造,同時保持輕量和高強度。這對於機器人的運動效率和續航能力至關重要。

快速原型設計:使用3D列印技術,設計師可以快速創建機器人的原型,並在實際應用之前進行測試和優化。

可持續性和環保:3D列印技術通常使用可再生材料,並且減少了浪費。這有助於降低機器人設計的環境影響。

\section{文獻回顧}
\subsection{Chat-GPT及其在自然語言處理中的應用}
引用Chat-GPT的相關論文並進一步討論

\subsection{Llama2模型的發展及其應用探索}
引用Llama2的相關論文並進一步討論

\subsection{運動學研究與機器人控制}
引用順向、逆向的相關論文並進一步討論,本實驗主要使用Policy-Gradient和基本三角函數(2 dof 機械臂)做你向運動學推導。

\subsection{carbon design的文獻}
引用carbon design的相關論文並進一步討論(用於設計3D模型)

\subsection{voronoi Diagram的文獻}
引用voronoi Diagram的相關論文並進一步討論(用於設計3D模型)

\end{document}