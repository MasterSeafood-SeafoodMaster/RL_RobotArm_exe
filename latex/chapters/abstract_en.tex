\documentclass[class=NCU_thesis, crop=false]{standalone}
\begin{document}

\chapter{Abstract}

This study explores how to use Large Language Models (LLMs) to translate natural language into code to control machines. The research encompasses background knowledge, literature review, research methods, experimental design, and results.

In the background knowledge and literature review section, the current state of research on large language models, intelligent machines, applications of AI in the Internet of Things, and the development status of 3D printing technology are introduced. Next, relevant literature on the application of automatic code generation in the field of machine control, kinematics research, robotic control, and recent developments in 3D printing are reviewed.

The research methods section details the hardware design process, including the model design software, file output formats, the use of 3D printing, and an introduction to motors and development boards. Regarding the software design process, the motion simulation environment, kinematics development, and the use of large language models and their official application programming interfaces are introduced. Finally, the system architecture chapter provides detailed descriptions of the system architecture diagram, system flowchart, and the overall program framework.

The experimental design and results section includes three experiments: basic control of a robotic arm, the application of the robotic arm in drawing, and the application of the robotic arm on an automatic transport vehicle. This section showcases mechanical design diagrams, function design, command formats, experiment process snapshots, and a summary of the final experimental outcomes.

The experimental results indicate that using large language models to generate code to control machines demonstrates considerable accuracy. Especially when clear machine function libraries are available, high-quality generation efficiency and accuracy can be achieved with minimal command input. However, with the increase in hardware and system complexity, some mechanical errors need to be addressed. Future research will continue to optimize system stability and accuracy, further enhancing its application value.

\vspace{2em}
\noindent \textbf{Keywords:} \keywordsEn{} % Set keywords in config.tex
\end{document}