\documentclass[class=NCU_thesis, crop=false]{standalone}
\begin{document}

\chapter{Abstract}

This study explores how to convert natural language into code to control machines using a Large Language Model (LLM). The research covers background knowledge, literature review, research methods, and experimental design and results.

In the background knowledge and literature review section, we first introduce the current research status of large language models, the application scenarios of smart machines and the Internet of Things (IoT) in artificial intelligence, and the development status of 3D printing technology. We then review the applications of large language models in code generation and machine control, kinematics research and robot control, as well as the application scenarios of 3D printing in smart machines.

The research methods section describes the hardware design process, including model design software, file output formats, the use of 3D printing, and the introduction of motors and development boards. The kinematics development aspect covers the development of a motion simulation environment, forward kinematics, and inverse kinematics. The section on the selection of large language models includes an introduction to LLM APIs and their applications. Finally, the system architecture section provides a detailed introduction to the system architecture diagram, system flowchart, and the overall program framework.

The experimental design and results section includes three experiments: basic control of the robotic arm, the application of the robotic arm in drawing, and the application of the robotic arm on an automated transport vehicle. This section presents mechanical structure design diagrams, function design, command format examples, and experimental results.

The experimental results show that using large language models to generate code for controlling machines achieves a high level of accuracy. Particularly with a well-defined machine function library, high-quality generation efficiency and accuracy can be achieved with minimal command input. However, as the number of hardware components increases and the system complexity grows, some mechanical errors have been encountered. Future research will continue to optimize the stability and accuracy of the system, further enhancing its application value.

\vspace{2em}
\noindent \textbf{Keywords:} \keywordsEn{} % Set keywords in config.tex
\end{document}