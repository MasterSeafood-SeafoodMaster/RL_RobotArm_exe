%%%%% build setting | 編譯設定 %%%%%
\setboolean{publish}{true} % {true}/{false} Set true before publish. 發佈前設true
\def\lang{zh} % {zh}/{en} Set main language.
\setboolean{disableChinese}{false} % {true}/{false} Disable Chinese, for English user who don't have Chinese fonts.

\synctex=1 % Enable SyncTeX.


%%%%% Information of your document. | 定義文件資訊 %%%%%
\def\deptshort {資訊工程}
\def\dept      {資 訊 工 程 學 系}%XX學系XXX碩士班(請參考「中央大學學位論文撰寫體例參考」附錄)
\def\degree    {碩 士 } % 碩士/博士
\def\titleZh   {使用大型語言模型進行機器控制指令的自動化生成}   % Chinese title
\def\titleEn   {Automated Generation of Machine Control Commands Using Large Language Models} % English title
\def\title     {\titleZh} % Main title, default Chinese title
\def\subtitle  {\titleEn} % Subtitle, default English title, empty allowed.
\def\logo      {}   % logo on titlepage. Watermark see below. 中央無校徽在封面。若你想加,樣板已附上,填入 ``logo-NCU.jpg'' 即可
\def\author    {蔡時富}
% \profs is professor's comma separated list. First is advisor , comma use ``{,}''
% \def\profs     {你的指導教授, 你的共同指導, ZUO{,} GONG-DE, 可再加更多人\dots}
\def\profs     {蘇木春}
\def\degreedate{中~華~民~國~一百一十三年~六~月}
\def\copyyear  {2022-2024}

\setboolean{printcopyright}{false} % {true}/{false} print copyright text on titlepage or cover.

\def\keywordsZh{自動程式碼生成, 深度學習, 大型語言模型, 機器控制, 自然語言}
\def\keywordsEn{Automatic Code Generation, Deep Learning, Large Language Models, Machine Control, Natural Language}


%%%%% Set letters filename | 設定插入文件 %%%%%
% This setting for main.tex default letters.
% It will not insert PDF if file non-exist or empty .
% 此為 main.tex 預設插入之 PDF,檔案不存在或空白則不插入。

% 碩博士論文電子檔授權書 Authorization Letter (for electronic)
\def\letterAuthEl{letters/letter_authorization.pdf}
% 碩博士紙本論文延後公開/下架申請書。(如需延後公開者,才需要裝訂於論文內頁)
\def\letterPubReq{letters/letter_publication_request.pdf}
% 指導教授推薦書
\def\letterRecom {letters/letter_recommendation.pdf}
% 口試委員審定書
\def\letterVerif {letters/letter_verification.pdf}
% 學術倫理修課證明
\def\letterCareCert {letters/letter_caree_certification.pdf}
% 論文相似度比對報告電子回條
\def\letterThesisSim {letters/letter_thesis_simularity.pdf}
% 遠距口試申請
\def\letterRemoteExam {letters/letter_remote_exam.pdf}

%%%%% Bibliography | 文獻列表 %%%%%
\def\bibManType{2} % {0}/{1}/{2} 0 = Embedded, 1 = BibTeX, 2 = biber / BibLaTeX
\def\bibStyle{ieee} % Default ``ieee'' with BibLaTeX. If you use BibTeX change to ``ieeetr''.
% BibLaTeX ref: https://www.sharelatex.com/learn/Biblatex_bibliography_styles
% BibTeX ref: https://www.sharelatex.com/learn/Bibtex_bibliography_styles
\setboolean{bibStyleNameYear}{false} % {true}/{false} true for use name,year to sort and cite.(If you want use custom option in .cls, set false here. )


%%%%% Set OS | 設定作業系統 %%%%%
% It will overwrite \OS if your LaTeX compile parameter has ``-shell-escape''(Texstudio default enable).
% Linux user: keep setting or add ``-shell-escape'' to compiler.
% Mac OS X user: Set to mac or add ``-shell-escape'' to compiler.
% Windows user: Don't need change setting.
\def\OS{win} % {linux}/{mac}/{win}, only effect auto select CJK font.(CJK means Chinese, Japanese, and Korean)


%%%%%%%%%%%%%%%%%%%%%%%%%%%%%%%%%%%%%%%%%%%%%%%%%%%%%%
%    Font size or style | 字體大小、風格
%%%%%%%%%%%%%%%%%%%%%%%%%%%%%%%%%%%%%%%%%%%%%%%%%%%%%%

%%%%% Set font | 設定中英文字型 %%%%%
% Keep empty for default font. CJK font must set OS for auto select.
% Setting by name of font(English will be better) or font filename.
% Linux 利用指令 fc-list :lang=zh 來查詢可以用的字體名稱。
% Windows 使用內建字型管理查詢。建議填英文字型名稱。
\def\mainfont   {Times New Roman} % default use Latin Modern Roman (lmodern pkg.)
\def\sansfont   {Times New Roman}
\def\monofont   {Times New Roman}
\def\CJKmainfont{DFKai-SB}
\def\CJKsansfont{DFKai-SB}
\def\CJKmonofont{DFKai-SB}


%%%%% Style of fonts and line stretch | 字體大小風格及行距 %%%%%
% Base font
\def\baseFontSize{14pt} % Valid: 8pt, 9pt, 10pt, 11pt, 12pt, 14pt, 17pt, 20pt
\def\baseLineStretch{1.5} % 行距(倍數)
\def\fakeBoldFactor{2} % 假粗體粗度。(Only affect CJK font.)
% Bibliography
\def\bibFontStyle{\small}
\def\bibLineStretch{1.2}
% Table
\def\floatFontStyle{\small} % content of table
\def\tableLineStretch{1.2}
% Caption
\def\captionFontStyle{\small}
\def\subcaptionFontStyle{\footnotesize}
\def\captionLineStretch{1.2}
% Page
\def\pageHeaderStyle{\rmfamily\CJKfamily{sf}\itshape\footnotesize}
\def\pageFooterStyle{\rmfamily\CJKfamily{sf}\footnotesize}
% listing font style set by basicstyle in \lstdefinestyle, see macro_preamble.tex
% todonotes font style set by textsize option when package loading, see thesis_base.cls . default \footnotesize


%%%%% Style of titles | 標題風格 %%%%%
\def\titlepageFontFamily{\rmfamily\CJKfamily{sf}}
% set "\rmfamily\CJKfamily{sf}" to mix rm for English and sf(default Kai) for CJK, see wiki on TW_Thesis_Template.
\def\abstractHeaderStyle{\rmfamily\CJKfamily{sf}\Large} % Information on abstract, cls default use \centering

\def\chapterTitleNumStyle   {\rmfamily\CJKfamily{sf}\LARGE\bfseries} % Number of chapter, only for en
\def\chapterTitleStyle      {\rmfamily\CJKfamily{sf}\huge\bfseries} % cls default use \centering for zh
\def\sectionTitleStyle      {\rmfamily\CJKfamily{sf}\Large\bfseries}
\def\subsectionTitleStyle   {\rmfamily\CJKfamily{sf}\large\bfseries}
\def\subsubsectionTitleStyle{\rmfamily\CJKfamily{sf}\normalsize\bfseries}
\def\paragraphTitleStyle    {\rmfamily\CJKfamily{sf}\normalsize\bfseries}
\def\subparagraphTitleStyle {\rmfamily\CJKfamily{sf}\normalsize\bfseries}

% Section numbering
\def\secNumDepth{4} % Depth of section Numbering. 設定章節標題給予數字標號的深度, \paragraph == 4
\def\titleNumStyle{0}              % 0/1 
    \def\indentBlockSSS{0mm}    % indent \subsubsection{} 
    \def\indentBlockPar{0mm}    % indent \paragraph{}
    \def\indentBlockSPar{0mm}   % indent \subparagraph{}


%%%%% Style of TOC | 目錄風格 %%%%%
\def\tocDepth{2}    % Depth of TOC. 目錄顯示層級,\subsection == 2
\def\tocStyleAlign{0}           % 0/1 see tutorial. (0 -> original)
\def\tocStyleChapter{1}         % 0/1/2
\def\tocStyleChapterFontFM{}    % {\itshape} etc. TOC chapter addition style to frontmatter. Only affect in \tocStyleChapter{1} .


%%%%%%%%%%%%%%%%%%%%%%%%%%%%%%%%%%%%%%%%%%%%%%%%%%%%%%
%    Pages style | 頁面風格
%%%%%%%%%%%%%%%%%%%%%%%%%%%%%%%%%%%%%%%%%%%%%%%%%%%%%%

%%%%% Watermark | 浮水印 %%%%%
% Add a watermark to every page after \startWatermark .
% It will search (image) filename in figures folder.

% Package (\wmMethod) bug / disadvantages:
% background: Background failed in all \includepdf page. Keep for compatibility.
% eso-pic: No, but still failed in some \includepdf page.
\def\wmContent{} % 圖檔名或文字 | image filename or text.(中央不用加,若你想加,樣板已附上,填入 ``logo-NCU.jpg'' 即可)
\def\wmMethod{1} % {0}/{1} 0 == background pkg. , 1 == eso-pic pkg. Recommend eso-pic
\def\wmScale{1}                 % 縮放比
\def\wmOpacity{1}               % 透明度
\def\wmShiftFromCenterX{0mm}    % 圖片中心相對紙張中心垂直位移
\def\wmShiftFromCenterY{0mm}    % 圖片中心相對紙張中心水平位移
\def\wmAngle{0}                 % 旋轉角度

% Start page:
% 0 == Manually, start at \startWatermark .
% 1 == Auto start at 1st physical page(include cover/titlepage).
% 2 == Auto start at 2nd physical page(include cover/titlepage).
% note: {1}/{2} request set ``publish'' (at line 2) to display.
\def\wmStartPage{2}             % {0}/{1}/{2}


%%%%% Really blank page | 純白空白頁 %%%%%
\setboolean{reallyBlankPage}{false} % {true}/{false} true for use  really blank pages between chapters. 


%%%%%%%%%%%%%%%%%%%%%%%%%%%%%%%%%%%%%%%%%%%%%%%%%%%%%%
%    Misc | 雜項
%%%%%%%%%%%%%%%%%%%%%%%%%%%%%%%%%%%%%%%%%%%%%%%%%%%%%%

%%%%% TOC (Table Of Contents) | 目錄 %%%%%
% Add TOC/LOF/LOT entry to TOC
\setboolean{tocEntryToToc}{true} % {true}/{false}, Table Of Contents
\setboolean{lofEntryToToc}{false} % {true}/{false}, List Of Figures
\setboolean{lotEntryToToc}{false} % {true}/{false}, List Of Tables


%%%%% PDF %%%%%
\setboolean{pdfLinkBoxDisplay}{true} % {true}/{false} ,Draw a box on the link. It only display in pdf viewer, not on paper.
